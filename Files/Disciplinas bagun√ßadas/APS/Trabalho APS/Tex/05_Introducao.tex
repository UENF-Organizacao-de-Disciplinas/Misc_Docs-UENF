\chapterimage{Pictures/Ausberto/1_sistemas.png} % Table of contents heading image

\chapter{Introdução}

    \begin{comment}
        Prof. Dr. Ausberto S. Castro Vera
        UENF - CCT - LCMAT - Curso de Ciência da Computação
        Campos, RJ,  2022
        Disciplina: Análise e Projeto de Sistemas
        Aluno: João Vítor Fernandes Dias
    \end{comment}

    %\textit{Análise e Projeto de Sistemas} é uma disciplina orientada a descrever as duas primeiras etapas do Ciclo de Vida de Desenvolvimento de um Sistema (CVDS), neste caso, um sistema computacional.  As referências bibliográficas básicas a serem consultadas são: \cite{Dennis2014}, \cite{Dennis2019} \cite{Gane1983} e \cite{Sommerville2011}. Como bibliografia complementar serão considerados: \cite{Satzinger2012}, \cite{Shelly2012}, \cite{Valacich2020}, \cite{Kendall2020}, \cite{Budgen2021} e \cite{Engholm2013}.

    Neste documento é apresentado, passo a passo,  as atividades relacionadas com a Análise e Design de um sistema acadêmico.

    \section{Descrição do Sistema Computacional a desenvolver}
        Inicialmente é preciso descrever qual o conceito geral no qual se baseia o projeto e isto será feito logo abaixo.
        
        \subsection{Pontos a aprimorar}
            Primeiramente, foi observada a vasta defasagem do sistema acadêmico atual em relação à tecnologia disponível no mercado. Para isso, uma das alternativas seria a reformulação do sistema, entretanto, as tecnologias utilizadas anteriormente poderiam servir como âncora para o desenvolvimento de um sistema acadêmico mais agradável ao usuário e que apresente eficácia, por isso foi vista a necessidade do desenvolvimento deste projeto que visa construir do zero um sistema acadêmico novo e melhor.
            
        \subsection{Pontos a inovar}
            Os pontos de inovação são características que não estavam presentes no sistema anterior, porém apresentariam significativa melhora no sistema como um todo. Um dos principais pontos visados é o cálculo de demanda de matérias pelos alunos. Outro dos pontos é a organização matemática da melhor disposição de professores, matérias e alunos no quadro de horários.

    \section{Identificando as componentes do meu sistema}
        Alguns componentes foram analisados como necessários pelo sistema:
        \subsection{Hardware}
            \begin{itemize}
                \item Ar condicionado
                \item Mesa
                \item Modem
                \item Mouse pad
                \item Pendrive
                \item Repetidor wi-fi
                \item Roteador
                \item Servidor
                \item Kit de informática completo
                \begin{itemize}
                    \item Computador
                    \item Monitor
                    \item Mouse
                    \item Teclado
                    \item Cabos
                    \begin{itemize}
                        \item hdmi-hdmi
                        \item de força ATX (2x)
                    \end{itemize}
                \end{itemize}
         	\end{itemize}
    
        \subsection{Software}
            \begin{itemize}
                \item Ambientes de desenvolvimento
                \item Antivírus
                \item Ferramentas de monitoramento do sistema
                \item Pacote office gratuito
                \item Sistema de banco de dados
                \item Sistema operacional gratuito para:
                \item Site do sistema acadêmico
                \begin{itemize}
                    \item Computadores
                    \item Servidor
                \end{itemize}
            \end{itemize}
    
        \subsection{Pessoas}
            \begin{itemize}
                \item Analista de banco de dados
                \item Coordenadoras de Centros Acadêmicos
                \item Coordenadoras de Curso
                \item Desenvolvedoras
                \item Gerente do projeto
                \item Secretária
            \end{itemize}
    
        \subsection{Banco de Dados}
            \begin{itemize}
                \item Base de dados:
                \begin{itemize}
                    \item Alunos
                    \begin{itemize}
                        \item Data de entrada
                        \item Email de login
                        \item Email institucional
                        \item Extrato com as notas das matérias
                        \item Matrícula
                        \item Nome
                        \item Nota do ENEM
                        \item Planos de estudo semestrais
                    \end{itemize}
                    \item Funcionários
                    \begin{itemize}
                        \item Professores
                        \begin{itemize}
                            \item Carga horária semanal
                            \item Código
                            \item Data de entrada
                            \item E-mail de login
                            \item E-mail institucional
                            \item Extrato com as notas das matérias ministradas
                            \item Link para o currículo Lattes
                            \item Nome
                            \item Nota de publicações acadêmicas
                            \item Planos de estudo semestrais
                            \item Salário
                            \item Vagas disponíveis para Iniciação Científica
                        \end{itemize}
                        \item Outros
                        \begin{itemize}
                            \item Código
                            \item Data de entrada
                            \item E-mail de login
                            \item E-mail institucional
                            \item Nome
                            \item Salário
                        \end{itemize}
                    \end{itemize}
                \end{itemize}
            \end{itemize}
    
        \subsection{Documentos}
            \begin{itemize}
                \item Formulários de:
                \begin{itemize}
                    \item Cancelamento de matrícula
                    \item Exclusão de matéria
                    \item Inclusão de matéria
                    \item Matrícula
                    \item Quebra de pré-requisito
                    \item Trancamento de matrícula
                    \item Transferência de curso
                \end{itemize}
            \end{itemize}
    
        \subsection{Metodologias ou Procedimentos}
            \begin{itemize}
                \item Levantamento de requisitos
                \begin{itemize}
                    \item Análise do sistema acadêmico atual
                    \item Listagem do que está sendo necessário
                    \item Listagem do que já se tem
                    \item Listagem do que não foi necessário
                \end{itemize}
                \item Formação de equipe
                \begin{itemize}
                    \item Apresentação da estrutura do sistema acadêmico
                    \item Tempo de adequação de um mês
                    \item Testes para averiguar o quanto aprenderam
                \end{itemize}
                \item Análise do sistema
                \begin{itemize}
                    \item Averiguar qual a demanda de processamento dos softwares
                    \item Desenvolvimento dos esquemas representativos do sistema
                \end{itemize}
                \item Implementação do sistema
                \begin{itemize}
                    \item Desenvolvimento do site
                    \begin{itemize}
                        \item Banco de dados
                        \item Back-end
                        \item Front-end
                    \end{itemize}
                    \item Montagem do servidor
                    \item Montagem da rede wi-fi
                    \item Montagem das salas com equipamentos de informática
                \end{itemize}
                \item Testes do sistema
                \begin{itemize}
                    \item Testes com professores
                    \item Testes com funcionários
                    \item Teste aberto
                    \item Teste de estresse
                    \item Teste de penetração
                \end{itemize}
                \item Capacitação para uso do sistema
                \begin{itemize}
                    \item Cursos de informática básica para todos os funcionários que interagirão com o sistema
                \end{itemize}
                \item Manutenção do sistema
                \begin{itemize}
                    \item Limpeza semanal na sala dos servidores
                    \item Limpeza diária nas salas com equipamentos de informática
                    \item Averiguações rotineiras quanto ao desempenho e bom funcionamento do sistema
                \end{itemize}
            \end{itemize}
    
        \subsection{Mobilidade}
            \begin{itemize}
                \item Telefones fixos conectados em rede interna
                \item Roteadores e repetidores com área de abrangência em todo o Campus
            \end{itemize}
    
        \subsection{Nuvem}
            \begin{itemize}
                \item Backup dos dados do servidor local
                \item Utilização de ferramentas para trabalho compartilhado como Google Docs e similares
            \end{itemize}
