\documentclass[twoside]{article}


\usepackage{amssymb}
\usepackage[table,xcdraw]{xcolor}
\usepackage{amsmath}
\usepackage[brazilian]{babel}
\usepackage[utf8x]{inputenc}
\usepackage[sc]{mathpazo}
\linespread{1.05}
\usepackage{microtype}
\usepackage[hang, small,labelfont=bf,up,textfont=it,up]{caption}
\usepackage{lettrine}
\usepackage{graphicx}

\usepackage[hmarginratio=1:1,top=20mm,bottom=35mm,right=15mm,left=15mm,columnsep=20pt]{geometry}
\usepackage{multicol}
\usepackage{booktabs}
\usepackage{float}
\usepackage{subfigure}
\usepackage{paralist}
\usepackage{hyperref}
\usepackage{abstract}
\renewcommand{\abstractnamefont}{\normalfont\bfseries}
\renewcommand{\abstracttextfont}{\normalfont\small\itshape}
\usepackage{titlesec}
\renewcommand\thesection{\Roman{section}}
\renewcommand\thesubsection{\Roman{subsection}}
\titleformat{\section}[block]{\large\scshape\centering}{\thesection.}{1em}{}
\titleformat{\subsection}[block]{\large}{\thesubsection.}{1em}{}

\usepackage{fancyhdr}
\pagestyle{fancy}
\fancyhead{}
\fancyfoot{}
%AQUI VOCE COLOCA A SIGL do grupo
\fancyhead[C]{Teoria Geral da Administração $\bullet$ Projeto de Pesquisa $\bullet$ \date{\today} }
\fancyfoot[RO,LE]{\thepage} \newenvironment{Figure}
  {\par\medskip\noindent\minipage{\linewidth}}
  {\endminipage\par\medskip}

% AQUI VOCÊ COLOCA O TÍTULO DO SEU EXPERIMENTO.

\title{\vspace{-10mm}\fontsize{18pt}{20pt}\selectfont\textbf{Sistemas Tecnológicos na Gestão do Sistema Único de Saúde (SUS)}}

% AQUI VOCE COLOCA O NOME DOS AUTORES DO TRABALHO

\author{
\large
\textsc{Daniel Terra Gomes,} \textsc{Estefânio Silva Ribeiro}\\[2mm]
%\textsc{Estefânio Silva Ribeiro}\\[2mm]
\large Universidade Estadual do Norte Fluminense Darcy Ribeiro – UENF\\[2mm]
\smallsize \href{mailto:danielterra@pq.uenf.br}{danielterra@pq.uenf.br}\\[2mm]
\smallsize \href{mailto:20201100011@pq.uenf.br}{20201100011@pq.uenf.br}\\[2mm]
%\smallsize\date{\today}
%\vspace{2mm}
}
\date{}

\begin{document}
\maketitle

\thispagestyle{fancy}

%___________________________________
%\raggedleft\date{\today} % Leave empty to omit a date

\begin{abstract}
  \textbf{}
\end{abstract}

\vspace{2mm}

\textbf{Palavras-chaves}: SUS, TI, Sistema Único de Saúde, Tecnologia, Dados, Sistemas.
%
\vspace{0.5cm}
\begin{multicols}{2}

  \section{Introdução}

  O Sistema Único de Saúde (SUS) inclui um sistema complexo de conexões de serviços públicos e privados. Foi idealizado em 1988 a partir do Movimento Sanitário Brasileiro, garantindo a saúde como direito do cidadão e dever do Estado, com a saúde pautada nos princípios da universalidade, equidade e integralidade \cite{Miranda_2017}.
  Essa idealização se torna ainda mais difícil. Sabendo que, o Brasil é um país extenso territorialmente e desigual economicamente.

  Dessa forma, o emprego de sistemas tecnológicos no setor de saúde têm permitido, no Brasil, agilizar e proporcionar mais inteligência nos atendimentos pelo SUS, visto que, pela alta demanda de atendimentos, uma forma de agilizar o processo faz-se mister, pois, dados mais atualizados mostram que em 2015 havia 1,95 médicos a cada mil habitantes. Porém, sabe-se que existem diversos problemas na aplicação real desta tecnologia da informação, desde falta de preparação dos funcionários à precariedade tecnológica dos pacientes, somados a uma má gestão que não só afeta o financeiro como também, diretamente, a saúde da população.

  Em 1990 tinha 1,12 médicos a cada mil habitantes, e em 2010 esse número foi para 1,86. O número ideal de médicos para uma população é uma contagem complexa. Tendo que ser levado em conta características como (demografia social, idade, gênero sexual); processos de trabalho (produtividade, carga de trabalho, serviço não clínico e variações no nível de atividade), características do sistema de saúde em vigor no país (por exemplo, cobertura e tipo
  dos serviços oferecidos), e as condições da população (socioeconômicas e epidemiológicas) \cite{Oliveira_2017}.

  Mormente, é fulcral ressaltar a importância da saúde para o ser vivo, e quando o assunto remete ao sistema de saúde, a prioridade consiste na qualidade, quantidade e eficácia dos atendimentos. De acordo com os dados do Mapa Assistencial, publicado pela Agência Nacional de Saúde Suplementar (ANS), foram realizados pelos SUS 1,57 bilhão de procedimentos como consultas, exames e internações somente no ano de 2018. Todavia, muitos desses atendimentos poderiam ser executados com maior eficiência através do uso de sistemas tecnológicos, visto que o uso dos mesmo facilitam o acesso às informações do paciente através de um banco de dados com todos os atendimentos que o cidadão recebeu em qualquer rede pública de saúde do Brasil. Esses dados proporcionam maior agilidade, onde por intermédio do mesmo, sabe-se se o paciente em questão tem algum tipo de alergia a medicamentos, histórico de atendimentos e etc.

  No entanto, como mencionado anteriormente, a execução dessa tecnologia está longe de ser perfeita. Os hospitais que têm acesso a essa tecnologia da informação não fazem o completo uso da mesma, onde atualmente utilizam-se apenas para registrar o atendimento dos pacientes. Apesar desta ser sua função primária, a gama de possibilidades que esse sistema fornece está além de apenas agilizar o atendimento e portabilizar os dados através da rede, possibilidades essas que afetam diretamente o lado financeiro, ambiental e da saúde \cite{Oliveira_2017}, \cite{Pinheiro_Filho_2012}.

  Algumas das possibilidades para aplicação do sistema de tecnologia, seriam:
  \begin{itemize}
    \item Solicitação de leitos: O hospital verifica no sistema se existem outros leitos nas proximidades, ou onde teria um mais próximo.
    \item Reabastecer estoque: Saber através de oferta e demanda, a quantidade que cada posto de saúde necessita. Visto que alguns hospitais têm demanda variada de remédios.
    \item Utilizar a demanda de remédios como medida para compra: Por meio da oferta e demanda dos remédios, o governo poderia saber quais remédios estão sendo mais solicitados que outros, e dar prioridade a esses, e diminuir a compra dos que têm pouca saída. Através desse método, os estoques terão os remédios sem falta e o dinheiro não será “desperdiçado” com remédios com baixa solicitação.
    \item Redução de lixo hospitalar: como citado, existem remédios com alta compra com baixa demanda, e quando eles não tem mais serventia por conta da validade ter vencido, esses remédios se tornam lixo hospitalar. Esse lixo hospitalar, que muita das vezes é mal conduzido, gera danos ambientais, que mais uma vez pesam no financeiro.
  \end{itemize}
  \begin{footnotesize}
    *Este estudo se propõe a analisar a organização do Sistema Único de Saúde no pacto federativo brasileiro, apontando para
    formas possíveis de garantir os princípios e diretrizes que a norteiam. Ademais, focando nos processos de gestão tendo o uso da tecnologia como principal ferramenta que busca uma igualdade ao acesso à rede de saúde.
  \end{footnotesize}

  \iffalse
    \section{Hipótese}



    %Exemplo Tabela 1

    \begin{table*}
      \centering
      \caption{Título da tabela}
      \begin{tabular}{cc c c c c}
        \hline
        Variável            & $g_i^1$ & $g_i^2$ & $g_i^3$ & $g_i^4$ & $g_i^5$ \\
        \hline
        $Re(\lambda)_{max}$ & -0.01   & -0.005  & -0.001  & -0.0005 & -0.0001 \\
        $u_{max}$           & 0.85    & 0.90    & 1       & 1.5     & 2       \\
        $t_{est}^{max}$     & 14      & 16      & 18      & 21      & 25      \\
        $noise_{max}$       & 0.5     & 0.9     & 1.2     & 1.4     & 1.5     \\
        $u_{nom}$           & 0.5     & 0.7     & 1       & 1.5     & 2       \\
        $t_{est}^{nom}$     & 10      & 11      & 12      & 14      & 15      \\
        \hline
      \end{tabular}
    \end{table*}

    Curabitur sed ante egestas, vehicula tortor quis, faucibus lorem. Nulla vel sollicitudin quam. Pellentesque non nunc at magna malesuada ullamcorper quis ut nunc. Suspendisse ac molestie turpis. Aliquam ac convallis augue. Praesent commodo, dolor at aliquam vehicula, augue nulla facilisis massa, sed pharetra urna nunc sit amet metus. Vestibulum sit amet molestie nibh, in congue urna. Curabitur condimentum sagittis consequat. Aliquam mollis eros nisl, in ultricies ante lacinia at. Orci varius

    %  \begin{figure}[H]
    %    \centering
    %    \includegraphics[scale=0.75]{figurateste.png}
    %    \caption{Legenda da figura.}
    %    \label{fig}
    %  \end{figure}


    \section{Justificativa}

    Phasellus ac iaculis neque. Cras ante magna, cursus eget consectetur id, rutrum sed libero. Fusce at nulla ut risus iaculis hendrerit at eu ex. Praesent pulvinar sem sed ligula gravida, a maximus leo sagittis. Integer efficitur eros eget tempus porttitor. Sed justo dolor,
    Proin et sem vitae ante porttitor viverra id et turpis. Aenean mollis eros vel ante tempor, ut elementum enim aliquam. Nam ut ante tempor, dapibus sem sit amet, dictum ipsum. Phasellus libero libero, vulputate nec iaculis quis, sagittis molestie magna. Morbi risus justo, ornare quis orci in, scelerisque convallis lacus. In congue pulvinar tincidunt. Aenean auctor lacus eu posuere sollicitudin. Maecenas a pulvinar ex, id aliquam tellus. Fusce scelerisque massa ex. Pellentesque placerat ac purus non ultricies. Maecenas sed feugiat nulla, vel pellentesque ipsum. Ut id imperdiet libero

    \section{----}

  \fi

  \section{Aspectos metodológicos}



  \section{Resultados e Discussão}




  \iffalse

    \begin{table*}[ht]
      \centering
      \caption{Legenda da tabela }
      \label{tabladeseables}
      \begin{tabular}{lcccccc}   \hline
        Controlador       & $Re(\lambda)_{max}$ & $u_{max}$   & $t_{est}^{max}$ & $noise_{max}$ & $u_{nom}$   & $t_{est}^{nom}$ \\ \hline
        B23               & INA                 & INA         & INA             & INA           & AD          & AIND            \\
        M23               & AD                  & AD          & AD              & T             & AD          & AIND            \\
        PPGA23            & \textbf{AD}         & \textbf{AD} & \textbf{AD}     & \textbf{AD}   & \textbf{AD} & \textbf{AD}     \\
        \hline
        W34               & AD                  & AD          & D               & T             & AD          & IND             \\
        M34               & AD                  & AD          & D               & AD            & AD          & AD              \\
        \textbf{PPGA23}*  & \textbf{AD}         & \textbf{AD} & \textbf{AD}     & \textbf{AD}   & \textbf{AD} & \textbf{AD}     \\
        \textbf{PPGA34}   & \textbf{AD}         & \textbf{AD} & \textbf{AD}     & \textbf{AD}   & \textbf{AD} & \textbf{AD}     \\
        \hline
        J45               & AD                  & IND         & AD              & IND           & AD          & AD              \\
        M45               & AD                  & AD          & IND             & T             & AD          & IND             \\
        \textbf{PPGA23}** & \textbf{D}          & \textbf{AD} & \textbf{D}      & \textbf{T}    & \textbf{AD} & \textbf{D}      \\
        \textbf{PPGA34}** & \textbf{AD}         & \textbf{AD} & \textbf{D}      & \textbf{D}    & \textbf{AD} & \textbf{D}      \\
        \textbf{PPGA45}   & \textbf{AD}         & \textbf{AD} & \textbf{AD}     & \textbf{AD}   & \textbf{AD} & \textbf{D}      \\
        \hline
      \end{tabular}
    \end{table*}


  \fi
  \begin{figure}[H]
    \centering
    \includegraphics[scale=0.35]{mil medicos a cade .png}
    \caption{Legenda da figura.}
    \label{fig2}
  \end{figure}

  \begin{figure}[H]
    \centering
    \includegraphics[scale=0.35]{medicos no sus.png}
    \caption{Legenda da figura.}
    \label{fig2}
  \end{figure}

  \begin{figure}[H]
    \centering
    \includegraphics[scale=0.35]{atencao basica.png}
    \caption{Legenda da figura.}
    \label{fig2}
  \end{figure}

  Dados: \cite{Viacava_2018}.





  \section{Conclusão}





  \bibliographystyle{plain} % We choose the &quot;plain&quot; reference style
  \bibliography{refs} % Entries are in the &quot;refs.bib&quot; file</code></pre> Entries are in the &quot;refs.bib&quot; file</code></pre>


\end{multicols}
\end{document}
