% ----------------------------------------------------------------
% Article Class (This is a LaTeX2e document)  ********************
% ----------------------------------------------------------------
\documentclass[12pt]{article}
\usepackage[brazil]{babel}
\usepackage{amsmath,amsthm}
\usepackage{amsfonts}
\usepackage{graphicx}
\usepackage{multirow}
\usepackage[top=2cm, bottom=2cm, left=2.5cm, right=2cm]{geometry}

\usepackage{datetime}
\newdateformat{DiaMesAno}{
  \monthname[\THEMONTH], \THEYEAR}

% ----------------------------------------------------------------
\begin{document}
\begin{tabular}{cl}
  \multirow{4}{*}{
    \includegraphics[width=2.3cm]{uenf.jpg} }
   & Curso: \textbf{Ci\^{e}ncia da Computa\c{c}\~{a}o} \\
   & Disciplina: \textbf{Geometria Anal\'{\i}tica}     \\
   & \textbf{Profa. Elba O. Bravo Asenjo}              \\
   & \DiaMesAno{\today}                                \\
\end{tabular}\\

\hrule width \hsize \kern 1mm \hrule width \hsize height 2pt



\begin{center}
  \huge\bf Lista de Exerc\'{\i}cios N$^{\circ}$ 3
\end{center}



% ----------------------------------------------------------------
\begin{enumerate}
  \item Em cada caso, determine o ponto $D$ tal que $CD \equiv AB$, onde $A=(-1,-1)$, $B=(2,3)$ e $C$ \'{e} o ponto:
        $(a)(2,1)$ ; \ \    $(b)(-2,0)$ ; \ \ $(c)(1,3)$.
  \item Determine o ponto $P$ tal que $\overrightarrow{OP} = \overrightarrow{AB}$, onde:
        \begin{enumerate}
          \item $A=(1,-1)$ \ \ e \ \ $B=(1,1)$
          \item $A=(-2,0)$ \ \  e \ \  $B=(1,3)$
        \end{enumerate}
  \item Sejam $A=(1,-1)$  e  $B=(4,1)$  v\'{e}rtices do paralelogramo  $P = ABDC$. Sabendo que as diagonais de $P$ se cortam no ponto $M = (3,2)$, encontre os v\'{e}rtices $C$  e  $D$.
  \item Considere os pontos $A = (-2,2)$, \ \ $B = (1,1)$, \ \ $C = (1,3)$, \ \ $D = (3,4)$, \ \ $E = (3,2)$, \ \ $F = (6,1)$, \ \ $G = (3,1)$, \ \ $H = (1,0)$. Ache os vetores, visualizando graficamente os vetores envolvidos no c\'{a}lculo:
        \begin{enumerate}
          \item $\overrightarrow{AB}+\overrightarrow{BC}+\overrightarrow{CD}$
          \item $2(\overrightarrow{BC} - \overrightarrow{EC}) + 3 \overrightarrow{EF}$
        \end{enumerate}

  \item Verifique que os vetores $\overrightarrow{u}$ \ e \ $\overrightarrow{v}$  n\~{a}o s\~{a}o m\'{u}ltiplos um do outro e escreva o vetor $\overrightarrow{w}$  como combina\c{c}\~{a}o linear de $\overrightarrow{u}$ \ e \  $\overrightarrow{v}$, onde:
        \begin{enumerate}
          \item  $\overrightarrow{u} = (1,1)$, \ \ $\overrightarrow{v} = (1,2)$ \ \ e \ \ $\overrightarrow{w} = (5,6)$;
          \item  $\overrightarrow{u} = (2,0)$, \ \ $\overrightarrow{v} = (2,2)$ \ \ e \ \ $\overrightarrow{w} = (0,1)$;
        \end{enumerate}
  \item Sejam $A =(-2,3,1)$,\ \ $B = (2,1,1)$, \ \ $C = (0,0,-3)$, \ \ $D = (10,-3,1)$, \ \ $E = (-1,2,3)$ \ \ e \ \ $F = (1,1,0)$ \ pontos do espa\c{c}o.
        \begin{enumerate}
          \item Determine o ponto $G$ tal que $AC \equiv DG$.
          \item Os pontos $E$, \ \ $F$ \ \ e \ \ $G$  s\~{a}o colineares? Justifique.
          \item Obtenha o ponto $H$ tal que $AB \equiv HD$.
          \item Os pontos $A$, \ \ $B$ \ \ e \ \ $H$  s\~{a}o colineares? Justifique.
        \end{enumerate}
  \item Mostre que os vetores $\overrightarrow{v_{1}}$, \ \ $\overrightarrow{v_{2}}$ \ \ e \ \ $\overrightarrow{v_{3}}$  s\~{a}o  linearmente independentes (LI) e escreva o vetor  $\overrightarrow{w} = (2,1,0)$  como combina\c{c}\~{a}o linear destes vetores, onde:
        \begin{enumerate}
          \item $\overrightarrow{v_{1}} = (1,1,0)$, \ \ $\overrightarrow{v_{2}} = (0,1,1)$ \ \ e \ \ $\overrightarrow{v_{3}} = (1,0,1)$;
          \item $\overrightarrow{v_{1}} = (1,1,1)$, \ \ $\overrightarrow{v_{2}} = (1,1,-1)$ \ \ e \ \ $\overrightarrow{v_{3}} = (1,-1,1)$.
        \end{enumerate}


\end{enumerate}



\end{document}
% ----------------------------------------------------------------
