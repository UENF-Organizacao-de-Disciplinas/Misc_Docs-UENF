\documentclass{article}
\usepackage{amsmath}
\usepackage{color,pxfonts,fix-cm}
\usepackage{latexsym}
\usepackage[mathletters]{ucs}
\DeclareUnicodeCharacter{46}{\textperiodcentered}
\DeclareUnicodeCharacter{8594}{$\rightarrow$}
\DeclareUnicodeCharacter{58}{$\colon$}
\DeclareUnicodeCharacter{8656}{$\Leftarrow$}
\DeclareUnicodeCharacter{124}{\textbar}
\DeclareUnicodeCharacter{960}{$\pi$}
\DeclareUnicodeCharacter{963}{$\sigma$}
\DeclareUnicodeCharacter{215}{$\times$}
\DeclareUnicodeCharacter{8704}{$\forall$}
\DeclareUnicodeCharacter{955}{$\lambda$}
\DeclareUnicodeCharacter{8712}{$\in$}
\DeclareUnicodeCharacter{8745}{$\cap$}
\DeclareUnicodeCharacter{8734}{$\infty$}
\DeclareUnicodeCharacter{8834}{$\subset$}
\DeclareUnicodeCharacter{8658}{$\Rightarrow$}
\DeclareUnicodeCharacter{8707}{$\exists$}
\usepackage[T1]{fontenc}
\usepackage[utf8x]{inputenc}
\usepackage{pict2e}
\usepackage{wasysym}
\usepackage[english]{babel}
\usepackage{tikz}
\pagestyle{empty}
\usepackage[margin=0in,paperwidth=595pt,paperheight=841pt]{geometry}
\begin{document}
\definecolor{color_29791}{rgb}{0,0,0}
\begin{tikzpicture}[overlay]\path(0pt,0pt);\end{tikzpicture}
\begin{picture}(-5,0)(2.5,0)
\put(186.848,-185.145){\fontsize{20.6625}{1}\usefont{T1}{cmr}{m}{n}\selectfont\color{color_29791}IPE: Lista 1}
\end{picture}
\begin{tikzpicture}[overlay]
\path(0pt,0pt);
\draw[color_29791,line width=0.398pt]
(230.974pt, -186.54pt) -- (287.966pt, -186.54pt)
;
\end{tikzpicture}
\begin{picture}(-5,0)(2.5,0)
\put(294.204,-185.145){\fontsize{20.6625}{1}\usefont{T1}{cmr}{m}{n}\selectfont\color{color_29791}/06-I I I-2023}
\end{picture}
\begin{tikzpicture}[overlay]
\path(0pt,0pt);
\draw[color_29791,line width=0.398pt]
(186.848pt, -191.688pt) -- (394.399pt, -191.688pt)
;
\end{tikzpicture}
\begin{picture}(-5,0)(2.5,0)
\put(188.36,-229.421){\fontsize{14.3462}{1}\usefont{T1}{cmr}{m}{n}\selectfont\color{color_29791}Ci ˆ encias da Computa¸ c˜ ao}
\end{picture}
\begin{tikzpicture}[overlay]
\path(0pt,0pt);
\draw[color_29791,line width=0.398pt]
(188.36pt, -233.605pt) -- (342.957pt, -233.605pt)
;
\end{tikzpicture}
\begin{picture}(-5,0)(2.5,0)
\put(342.957,-229.421){\fontsize{14.3462}{1}\usefont{T1}{cmr}{m}{n}\selectfont\color{color_29791}: UENF}
\put(246.924,-257.466){\fontsize{14.3462}{1}\usefont{T1}{cmr}{m}{n}\selectfont\color{color_29791}Marc h 7, 2023}
\put(110.66,-320.275){\fontsize{11.9552}{1}\usefont{T1}{cmr}{m}{n}\selectfont\color{color_29791}1. Q uan tas e quais s˜ ao as σ -´ algebras em}
\put(125.617,-339.106){\fontsize{11.9552}{1}\usefont{T1}{cmr}{m}{n}\selectfont\color{color_29791}(i) Ω = I I}
\put(175.252,-340.899){\fontsize{7.9701}{1}\usefont{T1}{cmr}{m}{n}\selectfont\color{color_29791}1}
\put(183.886,-339.106){\fontsize{11.9552}{1}\usefont{T1}{cmr}{m}{n}\selectfont\color{color_29791}?}
\put(125.617,-357.936){\fontsize{11.9552}{1}\usefont{T1}{cmr}{m}{n}\selectfont\color{color_29791}(ii) Ω = I I}
\put(178.503,-359.73){\fontsize{7.9701}{1}\usefont{T1}{cmr}{m}{n}\selectfont\color{color_29791}2}
\put(187.138,-357.936){\fontsize{11.9552}{1}\usefont{T1}{cmr}{m}{n}\selectfont\color{color_29791}?}
\put(125.617,-376.767){\fontsize{11.9552}{1}\usefont{T1}{cmr}{m}{n}\selectfont\color{color_29791}(iii) Ω = I I}
\put(181.755,-378.56){\fontsize{7.9701}{1}\usefont{T1}{cmr}{m}{n}\selectfont\color{color_29791}3}
\put(190.389,-376.767){\fontsize{11.9552}{1}\usefont{T1}{cmr}{m}{n}\selectfont\color{color_29791}?}
\put(110.66,-399.982){\fontsize{11.9552}{1}\usefont{T1}{cmr}{m}{n}\selectfont\color{color_29791}2. De monstrar as prop osi¸ c˜ oes 1), 2) e 3).}
\put(110.66,-423.198){\fontsize{11.9552}{1}\usefont{T1}{cmr}{m}{n}\selectfont\color{color_29791}3. S eja A = \{\{ 1 , 2 , 3 \} , \{ 4 , 5 \} , \{ \} , Ω \} ; onde Ω = \{ 1 , 2 , 3 , 4 , 5 \} . Seja (Ω , P (Ω) , P )}
\put(125.6169,-438.74){\fontsize{11.9552}{1}\usefont{T1}{cmr}{m}{n}\selectfont\color{color_29791}o espa¸ co de probabilidades usual, i.e., P ( A ) =}
\put(366.2059,-433.1609){\fontsize{7.9701}{1}\usefont{T1}{cmr}{m}{it}\selectfont\color{color_29791}| A |}
\end{picture}
\begin{tikzpicture}[overlay]
\path(0pt,0pt);
\draw[color_29791,line width=0.398pt]
(366.206pt, -435.751pt) -- (377.254pt, -435.751pt)
;
\end{tikzpicture}
\begin{picture}(-5,0)(2.5,0)
\put(366.319,-442.862){\fontsize{7.9701}{1}\usefont{T1}{cmr}{m}{it}\selectfont\color{color_29791}| Ω |}
\put(378.449,-438.74){\fontsize{11.9552}{1}\usefont{T1}{cmr}{m}{n}\selectfont\color{color_29791}.}
\put(137.714,-465.945){\fontsize{11.9552}{1}\usefont{T1}{cmr}{m}{n}\selectfont\color{color_29791}i)}
\put(152.426,-462.923){\fontsize{11.9552}{1}\usefont{T1}{cmr}{m}{n}\selectfont\color{color_29791}´}
\put(151.37,-465.945){\fontsize{11.9552}{1}\usefont{T1}{cmr}{m}{n}\selectfont\color{color_29791}E A uma σ -´ algebra de Ω. Justifique!}
\put(134.462,-484.179){\fontsize{11.9552}{1}\usefont{T1}{cmr}{m}{n}\selectfont\color{color_29791}ii)}
\put(152.426,-481.157){\fontsize{11.9552}{1}\usefont{T1}{cmr}{m}{n}\selectfont\color{color_29791}´}
\put(151.37,-484.179){\fontsize{11.9552}{1}\usefont{T1}{cmr}{m}{n}\selectfont\color{color_29791}E p oss ´ ıv el exibir uma σ -´ algebra de Ω com ap enas 3 elemen tos?}
\put(131.21,-502.413){\fontsize{11.9552}{1}\usefont{T1}{cmr}{m}{n}\selectfont\color{color_29791}iii) Escol her B ∈ P (Ω) tal que P ( B ) =}
\put(337.346,-497.706){\fontsize{7.9701}{1}\usefont{T1}{cmr}{m}{n}\selectfont\color{color_29791}2}
\end{picture}
\begin{tikzpicture}[overlay]
\path(0pt,0pt);
\draw[color_29791,line width=0.398pt]
(337.346pt, -499.424pt) -- (341.58pt, -499.424pt)
;
\end{tikzpicture}
\begin{picture}(-5,0)(2.5,0)
\put(337.346,-506.536){\fontsize{7.9701}{1}\usefont{T1}{cmr}{m}{n}\selectfont\color{color_29791}5}
\put(346.678,-502.413){\fontsize{11.9552}{1}\usefont{T1}{cmr}{m}{n}\selectfont\color{color_29791}e defina}
\put(209.516,-532.366){\fontsize{11.9552}{1}\usefont{T1}{cmr}{m}{it}\selectfont\color{color_29791}f}
\put(215.286,-534.1591){\fontsize{7.9701}{1}\usefont{T1}{cmr}{m}{it}\selectfont\color{color_29791}B}
\put(225.885,-532.366){\fontsize{11.9552}{1}\usefont{T1}{cmr}{m}{n}\selectfont\color{color_29791}: P (Ω) → I R , A 7→ f}
\put(330.372,-534.1591){\fontsize{7.9701}{1}\usefont{T1}{cmr}{m}{it}\selectfont\color{color_29791}B}
\put(337.65,-532.366){\fontsize{11.9552}{1}\usefont{T1}{cmr}{m}{n}\selectfont\color{color_29791}( A ) =}
\put(372.472,-524.278){\fontsize{11.9552}{1}\usefont{T1}{cmr}{m}{it}\selectfont\color{color_29791}P ( A ∩ B )}
\end{picture}
\begin{tikzpicture}[overlay]
\path(0pt,0pt);
\draw[color_29791,line width=0.398pt]
(372.472pt, -529.377pt) -- (422.303pt, -529.377pt)
;
\end{tikzpicture}
\begin{picture}(-5,0)(2.5,0)
\put(383.501,-540.566){\fontsize{11.9552}{1}\usefont{T1}{cmr}{m}{it}\selectfont\color{color_29791}P ( B )}
\put(423.498,-532.366){\fontsize{11.9552}{1}\usefont{T1}{cmr}{m}{it}\selectfont\color{color_29791}.}
\put(151.37,-561.297){\fontsize{11.9552}{1}\usefont{T1}{cmr}{m}{n}\selectfont\color{color_29791}Existem v´ arios sub c onjun tos A ⊂ Ω tais que f}
\put(400.038,-563.091){\fontsize{7.9701}{1}\usefont{T1}{cmr}{m}{it}\selectfont\color{color_29791}B}
\put(407.316,-561.297){\fontsize{11.9552}{1}\usefont{T1}{cmr}{m}{n}\selectfont\color{color_29791}( A ) = 1? (em}
\put(151.3701,-575.743){\fontsize{11.9552}{1}\usefont{T1}{cmr}{m}{n}\selectfont\color{color_29791}caso afirmat iv o, quan tos e quais s˜ ao s˜ ao estes ev en tos A ∈ P (Ω)?}
\put(151.3701,-590.189){\fontsize{11.9552}{1}\usefont{T1}{cmr}{m}{n}\selectfont\color{color_29791}) f}
\put(166.8951,-591.982){\fontsize{7.9701}{1}\usefont{T1}{cmr}{m}{it}\selectfont\color{color_29791}B}
\put(174.1741,-590.189){\fontsize{11.9552}{1}\usefont{T1}{cmr}{m}{n}\selectfont\color{color_29791}( A}
\put(187.5011,-591.982){\fontsize{7.9701}{1}\usefont{T1}{cmr}{m}{n}\selectfont\color{color_29791}1}
\put(194.8901,-590.189){\fontsize{11.9552}{1}\usefont{T1}{cmr}{m}{it}\selectfont\color{color_29791}∩ A}
\put(214.2921,-591.982){\fontsize{7.9701}{1}\usefont{T1}{cmr}{m}{n}\selectfont\color{color_29791}2}
\put(219.0251,-590.189){\fontsize{11.9552}{1}\usefont{T1}{cmr}{m}{n}\selectfont\color{color_29791}) = f}
\put(245.0931,-591.982){\fontsize{7.9701}{1}\usefont{T1}{cmr}{m}{it}\selectfont\color{color_29791}B}
\put(252.3721,-590.189){\fontsize{11.9552}{1}\usefont{T1}{cmr}{m}{n}\selectfont\color{color_29791}( A}
\put(265.6991,-591.982){\fontsize{7.9701}{1}\usefont{T1}{cmr}{m}{n}\selectfont\color{color_29791}1}
\put(270.4321,-590.189){\fontsize{11.9552}{1}\usefont{T1}{cmr}{m}{n}\selectfont\color{color_29791}) + f}
\put(295.1721,-591.982){\fontsize{7.9701}{1}\usefont{T1}{cmr}{m}{it}\selectfont\color{color_29791}B}
\put(302.4501,-590.189){\fontsize{11.9552}{1}\usefont{T1}{cmr}{m}{n}\selectfont\color{color_29791}( A}
\put(315.7781,-591.982){\fontsize{7.9701}{1}\usefont{T1}{cmr}{m}{n}\selectfont\color{color_29791}2}
\put(320.5101,-590.189){\fontsize{11.9552}{1}\usefont{T1}{cmr}{m}{n}\selectfont\color{color_29791}) ∀ A}
\put(343.8011,-591.982){\fontsize{7.9701}{1}\usefont{T1}{cmr}{m}{n}\selectfont\color{color_29791}1}
\put(348.5331,-590.189){\fontsize{11.9552}{1}\usefont{T1}{cmr}{m}{it}\selectfont\color{color_29791}, A}
\put(362.5521,-591.982){\fontsize{7.9701}{1}\usefont{T1}{cmr}{m}{n}\selectfont\color{color_29791}2}
\put(370.606,-590.189){\fontsize{11.9552}{1}\usefont{T1}{cmr}{m}{it}\selectfont\color{color_29791}⊂ Ω? f}
\put(408.18,-591.982){\fontsize{7.9701}{1}\usefont{T1}{cmr}{m}{it}\selectfont\color{color_29791}B}
\put(415.458,-590.189){\fontsize{11.9552}{1}\usefont{T1}{cmr}{m}{n}\selectfont\color{color_29791}(Ω) = 1?}
\put(110.66,-625.725){\fontsize{11.9552}{1}\usefont{T1}{cmr}{m}{n}\selectfont\color{color_29791}4. S eja A =}
\put(175.177,-611.179){\fontsize{9.9626}{1}\usefont{T1}{cmr}{m}{n}\selectfont\color{color_29791}?}
\put(184.503,-618.502){\fontsize{11.9552}{1}\usefont{T1}{cmr}{m}{it}\selectfont\color{color_29791}p ¯ q}
\put(185.489,-632.948){\fontsize{11.9552}{1}\usefont{T1}{cmr}{m}{n}\selectfont\color{color_29791}¯ p q}
\put(209.696,-611.179){\fontsize{9.9626}{1}\usefont{T1}{cmr}{m}{n}\selectfont\color{color_29791}?}
\put(220.931,-625.725){\fontsize{11.9552}{1}\usefont{T1}{cmr}{m}{n}\selectfont\color{color_29791}uma matriz Mark o viana, i.e.,}
\put(373.43,-604.803){\fontsize{9.9626}{1}\usefont{T1}{cmr}{m}{n}\selectfont\color{color_29791}}
\put(373.43,-613.77){\fontsize{9.9626}{1}\usefont{T1}{cmr}{m}{n}\selectfont\color{color_29791}}
\put(373.43,-631.703){\fontsize{9.9626}{1}\usefont{T1}{cmr}{m}{n}\selectfont\color{color_29791}}
\put(385.265,-610.6479){\fontsize{11.9552}{1}\usefont{T1}{cmr}{m}{n}\selectfont\color{color_29791}¯ p = 1 − p}
\put(385.137,-625.094){\fontsize{11.9552}{1}\usefont{T1}{cmr}{m}{n}\selectfont\color{color_29791}¯ q = 1 − q}
\put(384.278,-639.54){\fontsize{11.9552}{1}\usefont{T1}{cmr}{m}{it}\selectfont\color{color_29791}p, q ∈ [0 , 1] .}
\put(443.5289,-625.725){\fontsize{11.9552}{1}\usefont{T1}{cmr}{m}{n}\selectfont\color{color_29791}.}
\put(125.6169,-656.876){\fontsize{11.9552}{1}\usefont{T1}{cmr}{m}{n}\selectfont\color{color_29791}Seja f ( λ ) = d et( A − λI ).}
\put(287.697,-686.764){\fontsize{11.9552}{1}\usefont{T1}{cmr}{m}{n}\selectfont\color{color_29791}1}
\end{picture}
\newpage
\begin{tikzpicture}[overlay]\path(0pt,0pt);\end{tikzpicture}
\begin{picture}(-5,0)(2.5,0)
\put(137.714,-141.06){\fontsize{11.9552}{1}\usefont{T1}{cmr}{m}{n}\selectfont\color{color_29791}i) D .q. f ( λ ) = 0 ⇐ ⇒ λ}
\put(264.803,-136.722){\fontsize{7.9701}{1}\usefont{T1}{cmr}{m}{n}\selectfont\color{color_29791}2}
\put(272.192,-141.06){\fontsize{11.9552}{1}\usefont{T1}{cmr}{m}{it}\selectfont\color{color_29791}− σ}
\put(290.801,-142.854){\fontsize{7.9701}{1}\usefont{T1}{cmr}{m}{n}\selectfont\color{color_29791}1}
\put(295.533,-141.06){\fontsize{11.9552}{1}\usefont{T1}{cmr}{m}{it}\selectfont\color{color_29791}λ + σ}
\put(323.433,-142.854){\fontsize{7.9701}{1}\usefont{T1}{cmr}{m}{n}\selectfont\color{color_29791}2}
\put(331.486,-141.06){\fontsize{11.9552}{1}\usefont{T1}{cmr}{m}{n}\selectfont\color{color_29791}= 0; onde}
\put(384.883,-126.515){\fontsize{9.9626}{1}\usefont{T1}{cmr}{m}{n}\selectfont\color{color_29791}?}
\put(394.347,-133.838){\fontsize{11.9552}{1}\usefont{T1}{cmr}{m}{it}\selectfont\color{color_29791}σ}
\put(401.001,-135.631){\fontsize{7.9701}{1}\usefont{T1}{cmr}{m}{n}\selectfont\color{color_29791}1}
\put(409.054,-133.838){\fontsize{11.9552}{1}\usefont{T1}{cmr}{m}{n}\selectfont\color{color_29791}= tra¸ co( A )}
\put(394.347,-148.283){\fontsize{11.9552}{1}\usefont{T1}{cmr}{m}{it}\selectfont\color{color_29791}σ}
\put(401.001,-150.077){\fontsize{7.9701}{1}\usefont{T1}{cmr}{m}{n}\selectfont\color{color_29791}2}
\put(409.054,-148.283){\fontsize{11.9552}{1}\usefont{T1}{cmr}{m}{n}\selectfont\color{color_29791}= det( A ) .}
\put(134.462,-172.808){\fontsize{11.9552}{1}\usefont{T1}{cmr}{m}{n}\selectfont\color{color_29791}ii) De monstrar que}
\put(237.44,-158.262){\fontsize{9.9626}{1}\usefont{T1}{cmr}{m}{n}\selectfont\color{color_29791}?}
\put(246.904,-164.954){\fontsize{11.9552}{1}\usefont{T1}{cmr}{m}{it}\selectfont\color{color_29791}σ}
\put(253.558,-166.7471){\fontsize{7.9701}{1}\usefont{T1}{cmr}{m}{n}\selectfont\color{color_29791}1}
\put(260.947,-164.954){\fontsize{11.9552}{1}\usefont{T1}{cmr}{m}{it}\selectfont\color{color_29791}− σ}
\put(279.5549,-166.7471){\fontsize{7.9701}{1}\usefont{T1}{cmr}{m}{n}\selectfont\color{color_29791}2}
\put(287.6089,-164.954){\fontsize{11.9552}{1}\usefont{T1}{cmr}{m}{n}\selectfont\color{color_29791}= 1 ,}
\put(246.9039,-179.4){\fontsize{11.9552}{1}\usefont{T1}{cmr}{m}{it}\selectfont\color{color_29791}f (1) = 0 = f ( σ}
\put(324.5059,-181.1931){\fontsize{7.9701}{1}\usefont{T1}{cmr}{m}{n}\selectfont\color{color_29791}2}
\put(329.2389,-179.4){\fontsize{11.9552}{1}\usefont{T1}{cmr}{m}{n}\selectfont\color{color_29791}) .}
\put(131.2099,-197.333){\fontsize{11.9552}{1}\usefont{T1}{cmr}{m}{n}\selectfont\color{color_29791}iii) Enc on trar um v etor de probabilidades π tal que Aπ = π .}
\put(131.535,-216.76){\fontsize{11.9552}{1}\usefont{T1}{cmr}{m}{n}\selectfont\color{color_29791}iv) Ac har uma matriz n˜ ao singular Q que diagonalize A .}
\put(134.787,-236.187){\fontsize{11.9552}{1}\usefont{T1}{cmr}{m}{n}\selectfont\color{color_29791}v) Enc on trar uma f´ orm ula, em termos da matriz diagonalizan te Q e}
\put(151.37,-250.633){\fontsize{11.9552}{1}\usefont{T1}{cmr}{m}{n}\selectfont\color{color_29791}a ma triz diagonalizada D , par a calcular A}
\put(372.984,-246.294){\fontsize{7.9701}{1}\usefont{T1}{cmr}{m}{it}\selectfont\color{color_29791}n}
\put(378.62,-250.633){\fontsize{11.9552}{1}\usefont{T1}{cmr}{m}{n}\selectfont\color{color_29791}, a n - ´ esima p ot ˆ encia}
\put(151.37,-265.079){\fontsize{11.9552}{1}\usefont{T1}{cmr}{m}{n}\selectfont\color{color_29791}da matriz Mark o viana A .}
\put(110.66,-292.227){\fontsize{11.9552}{1}\usefont{T1}{cmr}{m}{n}\selectfont\color{color_29791}5. S ejam as matrizes A =}
\put(249.506,-287.52){\fontsize{7.9701}{1}\usefont{T1}{cmr}{m}{n}\selectfont\color{color_29791}1}
\end{picture}
\begin{tikzpicture}[overlay]
\path(0pt,0pt);
\draw[color_29791,line width=0.398pt]
(247.389pt, -289.238pt) -- (255.857pt, -289.238pt)
;
\end{tikzpicture}
\begin{picture}(-5,0)(2.5,0)
\put(247.389,-296.349){\fontsize{7.9701}{1}\usefont{T1}{cmr}{m}{n}\selectfont\color{color_29791}15}
\put(259.045,-277.681){\fontsize{9.9626}{1}\usefont{T1}{cmr}{m}{n}\selectfont\color{color_29791}?}
\put(271.298,-285.004){\fontsize{11.9552}{1}\usefont{T1}{cmr}{m}{n}\selectfont\color{color_29791}5 12}
\put(268.371,-299.45){\fontsize{11.9552}{1}\usefont{T1}{cmr}{m}{n}\selectfont\color{color_29791}10 3}
\put(305.482,-277.681){\fontsize{9.9626}{1}\usefont{T1}{cmr}{m}{n}\selectfont\color{color_29791}?}
\put(312.815,-292.2271){\fontsize{11.9552}{1}\usefont{T1}{cmr}{m}{n}\selectfont\color{color_29791}, Q n˜ ao singular e D diagonal.}
\put(137.714,-322.872){\fontsize{11.9552}{1}\usefont{T1}{cmr}{m}{n}\selectfont\color{color_29791}i) ∃ Q / Q}
\put(189.049,-318.5341){\fontsize{7.9701}{1}\usefont{T1}{cmr}{m}{it}\selectfont\color{color_29791}− 1}
\put(200.368,-322.8721){\fontsize{11.9552}{1}\usefont{T1}{cmr}{m}{it}\selectfont\color{color_29791}AQ = D [em caso afirma tiv o, exibi r uma tal matriz Q ]}
\put(134.462,-342.2991){\fontsize{11.9552}{1}\usefont{T1}{cmr}{m}{n}\selectfont\color{color_29791}ii) S e Q diagonaliza A em D , exibir uma f´ orm ula para A}
\put(414.487,-337.9611){\fontsize{7.9701}{1}\usefont{T1}{cmr}{m}{it}\selectfont\color{color_29791}n}
\put(423.445,-342.2991){\fontsize{11.9552}{1}\usefont{T1}{cmr}{m}{n}\selectfont\color{color_29791}= A × . . . × A}
\put(435.87,-347.4801){\fontsize{9.9626}{1}\usefont{T1}{cmr}{m}{n}\selectfont\color{color_29791}|}
\end{picture}
\begin{tikzpicture}[overlay]
\path(0pt,0pt);
\filldraw[color_29791][nonzero rule]
(440.353pt, -347.48pt) -- (461.644pt, -347.48pt)
 -- (461.644pt, -347.48pt)
 -- (461.644pt, -346.285pt)
 -- (461.644pt, -346.285pt)
 -- (440.353pt, -346.285pt) -- cycle
;
\end{tikzpicture}
\begin{picture}(-5,0)(2.5,0)
\put(461.644,-347.48){\fontsize{9.9626}{1}\usefont{T1}{cmr}{m}{n}\selectfont\color{color_29791}\{z}
\end{picture}
\begin{tikzpicture}[overlay]
\path(0pt,0pt);
\filldraw[color_29791][nonzero rule]
(470.61pt, -347.48pt) -- (491.901pt, -347.48pt)
 -- (491.901pt, -347.48pt)
 -- (491.901pt, -346.285pt)
 -- (491.901pt, -346.285pt)
 -- (470.61pt, -346.285pt) -- cycle
;
\end{tikzpicture}
\begin{picture}(-5,0)(2.5,0)
\put(491.901,-347.48){\fontsize{9.9626}{1}\usefont{T1}{cmr}{m}{n}\selectfont\color{color_29791}\}}
\put(439.381,-360.431){\fontsize{7.9701}{1}\usefont{T1}{cmr}{m}{it}\selectfont\color{color_29791}n fatores A}
\put(496.384,-342.299){\fontsize{11.9552}{1}\usefont{T1}{cmr}{m}{n}\selectfont\color{color_29791}.}
\put(131.21,-383.594){\fontsize{11.9552}{1}\usefont{T1}{cmr}{m}{n}\selectfont\color{color_29791}iii) Ob ter lim}
\put(201.446,-385.388){\fontsize{7.9701}{1}\usefont{T1}{cmr}{m}{it}\selectfont\color{color_29791}n → + ∞}
\put(232.598,-383.594){\fontsize{11.9552}{1}\usefont{T1}{cmr}{m}{it}\selectfont\color{color_29791}A}
\put(241.373,-379.256){\fontsize{7.9701}{1}\usefont{T1}{cmr}{m}{it}\selectfont\color{color_29791}n}
\put(250.331,-383.594){\fontsize{11.9552}{1}\usefont{T1}{cmr}{m}{n}\selectfont\color{color_29791}=}
\put(262.756,-369.049){\fontsize{9.9626}{1}\usefont{T1}{cmr}{m}{n}\selectfont\color{color_29791}?}
\put(272.082,-376.372){\fontsize{11.9552}{1}\usefont{T1}{cmr}{m}{it}\selectfont\color{color_29791}a a}
\put(272.666,-390.817){\fontsize{11.9552}{1}\usefont{T1}{cmr}{m}{it}\selectfont\color{color_29791}b b}
\put(298.071,-369.049){\fontsize{9.9626}{1}\usefont{T1}{cmr}{m}{n}\selectfont\color{color_29791}?}
\put(305.404,-383.594){\fontsize{11.9552}{1}\usefont{T1}{cmr}{m}{n}\selectfont\color{color_29791}; onde}
\put(340.522,-369.049){\fontsize{9.9626}{1}\usefont{T1}{cmr}{m}{n}\selectfont\color{color_29791}?}
\put(349.848,-376.372){\fontsize{11.9552}{1}\usefont{T1}{cmr}{m}{it}\selectfont\color{color_29791}a}
\put(350.432,-390.817){\fontsize{11.9552}{1}\usefont{T1}{cmr}{m}{it}\selectfont\color{color_29791}b}
\put(357.986,-369.049){\fontsize{9.9626}{1}\usefont{T1}{cmr}{m}{n}\selectfont\color{color_29791}?}
\put(368.64,-383.594){\fontsize{11.9552}{1}\usefont{T1}{cmr}{m}{n}\selectfont\color{color_29791}=}
\put(384.378,-378.887){\fontsize{7.9701}{1}\usefont{T1}{cmr}{m}{n}\selectfont\color{color_29791}6}
\end{picture}
\begin{tikzpicture}[overlay]
\path(0pt,0pt);
\draw[color_29791,line width=0.398pt]
(382.261pt, -380.606pt) -- (390.729pt, -380.606pt)
;
\end{tikzpicture}
\begin{picture}(-5,0)(2.5,0)
\put(382.261,-387.717){\fontsize{7.9701}{1}\usefont{T1}{cmr}{m}{n}\selectfont\color{color_29791}11}
\put(393.918,-369.049){\fontsize{9.9626}{1}\usefont{T1}{cmr}{m}{n}\selectfont\color{color_29791}?}
\put(403.244,-376.372){\fontsize{11.9552}{1}\usefont{T1}{cmr}{m}{n}\selectfont\color{color_29791}6}
\put(403.244,-390.817){\fontsize{11.9552}{1}\usefont{T1}{cmr}{m}{n}\selectfont\color{color_29791}5}
\put(411.089,-369.049){\fontsize{9.9626}{1}\usefont{T1}{cmr}{m}{n}\selectfont\color{color_29791}?}
\put(418.423,-383.594){\fontsize{11.9552}{1}\usefont{T1}{cmr}{m}{n}\selectfont\color{color_29791}.}
\put(110.66,-413.731){\fontsize{11.9552}{1}\usefont{T1}{cmr}{m}{n}\selectfont\color{color_29791}6. S eja ( X}
\put(166.209,-415.525){\fontsize{7.9701}{1}\usefont{T1}{cmr}{m}{it}\selectfont\color{color_29791}n}
\put(171.846,-413.731){\fontsize{11.9552}{1}\usefont{T1}{cmr}{m}{n}\selectfont\color{color_29791})}
\put(176.398,-417.783){\fontsize{7.9701}{1}\usefont{T1}{cmr}{m}{it}\selectfont\color{color_29791}n ∈ I N}
\put(208.556,-413.731){\fontsize{11.9552}{1}\usefont{T1}{cmr}{m}{n}\selectfont\color{color_29791}um pro cesso de Mark o v, onde X}
\put(381.233,-415.525){\fontsize{7.9701}{1}\usefont{T1}{cmr}{m}{it}\selectfont\color{color_29791}n}
\put(392.38,-413.731){\fontsize{11.9552}{1}\usefont{T1}{cmr}{m}{n}\selectfont\color{color_29791}: Ω → S = I I}
\put(469.772,-415.525){\fontsize{7.9701}{1}\usefont{T1}{cmr}{m}{n}\selectfont\color{color_29791}3}
\put(479.693,-413.731){\fontsize{11.9552}{1}\usefont{T1}{cmr}{m}{n}\selectfont\color{color_29791}e}
\put(125.617,-428.177){\fontsize{11.9552}{1}\usefont{T1}{cmr}{m}{n}\selectfont\color{color_29791}cuja matriz de transi¸ c˜ ao de probabilidades P = ( p}
\put(379.001,-429.971){\fontsize{7.9701}{1}\usefont{T1}{cmr}{m}{it}\selectfont\color{color_29791}ij}
\put(386.266,-428.177){\fontsize{11.9552}{1}\usefont{T1}{cmr}{m}{n}\selectfont\color{color_29791})}
\put(390.819,-429.971){\fontsize{7.9701}{1}\usefont{T1}{cmr}{m}{n}\selectfont\color{color_29791}3 × 3}
\put(409.958,-428.177){\fontsize{11.9552}{1}\usefont{T1}{cmr}{m}{n}\selectfont\color{color_29791}tem as colunas}
\put(125.617,-456.427){\fontsize{11.9552}{1}\usefont{T1}{cmr}{m}{it}\selectfont\color{color_29791}P}
\put(134.788,-452.089){\fontsize{7.9701}{1}\usefont{T1}{cmr}{m}{n}\selectfont\color{color_29791}[1]}
\put(147.546,-456.427){\fontsize{11.9552}{1}\usefont{T1}{cmr}{m}{n}\selectfont\color{color_29791}=}
\put(161.167,-451.72){\fontsize{7.9701}{1}\usefont{T1}{cmr}{m}{n}\selectfont\color{color_29791}1}
\end{picture}
\begin{tikzpicture}[overlay]
\path(0pt,0pt);
\draw[color_29791,line width=0.398pt]
(161.167pt, -453.438pt) -- (165.401pt, -453.438pt)
;
\end{tikzpicture}
\begin{picture}(-5,0)(2.5,0)
\put(161.167,-460.55){\fontsize{7.9701}{1}\usefont{T1}{cmr}{m}{n}\selectfont\color{color_29791}3}
\put(168.589,-432.915){\fontsize{9.9626}{1}\usefont{T1}{cmr}{m}{n}\selectfont\color{color_29791}}
\put(168.589,-450.449){\fontsize{9.9626}{1}\usefont{T1}{cmr}{m}{n}\selectfont\color{color_29791}}
\put(168.589,-456.826){\fontsize{9.9626}{1}\usefont{T1}{cmr}{m}{n}\selectfont\color{color_29791}}
\put(179.299,-441.981){\fontsize{11.9552}{1}\usefont{T1}{cmr}{m}{n}\selectfont\color{color_29791}1}
\put(179.299,-456.427){\fontsize{11.9552}{1}\usefont{T1}{cmr}{m}{n}\selectfont\color{color_29791}1}
\put(179.299,-470.873){\fontsize{11.9552}{1}\usefont{T1}{cmr}{m}{n}\selectfont\color{color_29791}1}
\put(187.145,-432.915){\fontsize{9.9626}{1}\usefont{T1}{cmr}{m}{n}\selectfont\color{color_29791}}
\put(187.145,-450.449){\fontsize{9.9626}{1}\usefont{T1}{cmr}{m}{n}\selectfont\color{color_29791}}
\put(187.145,-456.826){\fontsize{9.9626}{1}\usefont{T1}{cmr}{m}{n}\selectfont\color{color_29791}}
\put(197.855,-456.427){\fontsize{11.9552}{1}\usefont{T1}{cmr}{m}{it}\selectfont\color{color_29791}, P}
\put(212.27,-452.089){\fontsize{7.9701}{1}\usefont{T1}{cmr}{m}{n}\selectfont\color{color_29791}[2]}
\put(225.028,-456.427){\fontsize{11.9552}{1}\usefont{T1}{cmr}{m}{n}\selectfont\color{color_29791}=}
\put(238.649,-451.72){\fontsize{7.9701}{1}\usefont{T1}{cmr}{m}{n}\selectfont\color{color_29791}1}
\end{picture}
\begin{tikzpicture}[overlay]
\path(0pt,0pt);
\draw[color_29791,line width=0.398pt]
(238.649pt, -453.438pt) -- (242.883pt, -453.438pt)
;
\end{tikzpicture}
\begin{picture}(-5,0)(2.5,0)
\put(238.649,-460.55){\fontsize{7.9701}{1}\usefont{T1}{cmr}{m}{n}\selectfont\color{color_29791}7}
\put(246.071,-432.915){\fontsize{9.9626}{1}\usefont{T1}{cmr}{m}{n}\selectfont\color{color_29791}}
\put(246.071,-450.449){\fontsize{9.9626}{1}\usefont{T1}{cmr}{m}{n}\selectfont\color{color_29791}}
\put(246.071,-456.826){\fontsize{9.9626}{1}\usefont{T1}{cmr}{m}{n}\selectfont\color{color_29791}}
\put(256.781,-441.981){\fontsize{11.9552}{1}\usefont{T1}{cmr}{m}{n}\selectfont\color{color_29791}4}
\put(256.781,-456.427){\fontsize{11.9552}{1}\usefont{T1}{cmr}{m}{n}\selectfont\color{color_29791}2}
\put(256.781,-470.873){\fontsize{11.9552}{1}\usefont{T1}{cmr}{m}{n}\selectfont\color{color_29791}1}
\put(264.626,-432.915){\fontsize{9.9626}{1}\usefont{T1}{cmr}{m}{n}\selectfont\color{color_29791}}
\put(264.626,-450.449){\fontsize{9.9626}{1}\usefont{T1}{cmr}{m}{n}\selectfont\color{color_29791}}
\put(264.626,-456.826){\fontsize{9.9626}{1}\usefont{T1}{cmr}{m}{n}\selectfont\color{color_29791}}
\put(275.336,-456.427){\fontsize{11.9552}{1}\usefont{T1}{cmr}{m}{it}\selectfont\color{color_29791}, P}
\put(289.751,-452.089){\fontsize{7.9701}{1}\usefont{T1}{cmr}{m}{n}\selectfont\color{color_29791}[3]}
\put(302.509,-456.427){\fontsize{11.9552}{1}\usefont{T1}{cmr}{m}{n}\selectfont\color{color_29791}=}
\put(316.13,-451.72){\fontsize{7.9701}{1}\usefont{T1}{cmr}{m}{n}\selectfont\color{color_29791}1}
\end{picture}
\begin{tikzpicture}[overlay]
\path(0pt,0pt);
\draw[color_29791,line width=0.398pt]
(316.13pt, -453.438pt) -- (320.364pt, -453.438pt)
;
\end{tikzpicture}
\begin{picture}(-5,0)(2.5,0)
\put(316.13,-460.55){\fontsize{7.9701}{1}\usefont{T1}{cmr}{m}{n}\selectfont\color{color_29791}5}
\put(323.552,-432.915){\fontsize{9.9626}{1}\usefont{T1}{cmr}{m}{n}\selectfont\color{color_29791}}
\put(323.552,-450.449){\fontsize{9.9626}{1}\usefont{T1}{cmr}{m}{n}\selectfont\color{color_29791}}
\put(323.552,-456.826){\fontsize{9.9626}{1}\usefont{T1}{cmr}{m}{n}\selectfont\color{color_29791}}
\put(334.262,-441.981){\fontsize{11.9552}{1}\usefont{T1}{cmr}{m}{n}\selectfont\color{color_29791}2}
\put(334.262,-456.427){\fontsize{11.9552}{1}\usefont{T1}{cmr}{m}{n}\selectfont\color{color_29791}2}
\put(334.262,-470.873){\fontsize{11.9552}{1}\usefont{T1}{cmr}{m}{n}\selectfont\color{color_29791}1}
\put(342.107,-432.915){\fontsize{9.9626}{1}\usefont{T1}{cmr}{m}{n}\selectfont\color{color_29791}}
\put(342.107,-450.449){\fontsize{9.9626}{1}\usefont{T1}{cmr}{m}{n}\selectfont\color{color_29791}}
\put(342.107,-456.826){\fontsize{9.9626}{1}\usefont{T1}{cmr}{m}{n}\selectfont\color{color_29791}}
\put(350.825,-456.427){\fontsize{11.9552}{1}\usefont{T1}{cmr}{m}{n}\selectfont\color{color_29791}; resp e ctiv amen te.}
\put(125.617,-489.304){\fontsize{11.9552}{1}\usefont{T1}{cmr}{m}{n}\selectfont\color{color_29791}i)Demonstrar que 1 ´ e um auto v alor de P .}
\put(125.617,-508.731){\fontsize{11.9552}{1}\usefont{T1}{cmr}{m}{n}\selectfont\color{color_29791}(ii) Ac har um v etor de probabilidades π / P π = π .}
\put(287.697,-686.764){\fontsize{11.9552}{1}\usefont{T1}{cmr}{m}{n}\selectfont\color{color_29791}2}
\end{picture}
\end{document}